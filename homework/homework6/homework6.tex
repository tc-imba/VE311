\documentclass{article}
\usepackage{enumerate}
\usepackage{amsmath}
\usepackage{amssymb}
\usepackage{graphicx}
\usepackage{subfigure}
\usepackage{geometry}
\usepackage{caption}
\usepackage{indentfirst}

\usepackage{tikz}
\usetikzlibrary{circuits.ee.IEC}
\usetikzlibrary{arrows.meta}
\usetikzlibrary{calc}

\usepackage{minted}
\newcommand{\inputmintedindent}[2]{
\begin{minipage}{0.1\linewidth}\end{minipage}
\begin{minipage}{0.85\linewidth}\inputminted{#1}{#2}\end{minipage}\\[0.5em]
}

\geometry{left=3.0cm,right=3.0cm,top=3.0cm,bottom=4.0cm}
\renewcommand{\thesection}{Problem \arabic{section}.}
%\allowdisplaybreaks[4]
\newcommand{\Omegacm}{{\rm\,\Omega\cdot cm}}
\newcommand{\unit}[1]{{\rm\,#1}}

\title{VE311 Homework 6}
\author{Liu Yihao 515370910207}
\date{}

\begin{document}
\maketitle

\section{}

For the common-mode gain, I wrote this SPICE circuit: \\

\inputmintedindent{v}{p1_1.cir}

The result of common-mode gain is $4.86588\times10^{-1}$. \\

For the differential-mode gain, I wrote this SPICE circuit: \\

\inputmintedindent{v}{p1_2.cir}

The result of differential-mode gain is $1.81396\times10^{1}$. 

$$CMRR = \frac{1.81396\times10^{1}}{2\cdot 4.86588\times10^{-1}} = 1.86395\times10^{1}$$

The differential-mode and common-mode input resistances are both $+\infty$.\\


\section{}

\begin{enumerate}[(a)]
\item
Suppose $V_{BE}=0.7\unit{V}$,

$$I_E=\frac{V_{EE}-V_{BE}}{2R_{EE}}=\frac{18\unit{V}-0.7\unit{V}}{2\cdot47\unit{k\Omega}}\approx0.184\unit{mA}$$
$$I_C=I_E\frac{\beta_F}{\beta_F+1}=0.368\unit{mA}\cdot\frac{100}{101}\approx0.182\unit{mA}$$
$$V_{C}=V_{CC}-I_CR_C=18\unit{V}-0.364\unit{mA}\cdot50\unit{k\Omega}\approx8.9\unit{V}$$
$$V_{CE}=V_{C}+V_{BE}=9.6\unit{V}$$

So the Q point is $(0.184\unit{mA},\ 9.6\unit{V})$.
\item
$$g_m=\frac{I_C}{V_T}=\frac{0.182\unit{mA}}{0.025\unit{V}}=7.28\unit{mS}$$
$$A_{dd}=-g_mR_C=-7.28\unit{mS}\cdot50\unit{k\Omega}=364$$
$$A_{cc}=-\frac{R_C}{2R_{EE}}=-\frac{50\unit{k\Omega}}{2\cdot47\unit{k\Omega}}\approx0.53$$
$$CMRR = \frac{A_{dd}}{2A_{cc}} = \frac{364}{2\cdot 0.53} \approx 343$$

Differential-mode:
$$r_{id}=2r_{\pi}=2\frac{\beta_F}{g_m}=\frac{2\cdot100}{7.28\unit{mS}}\approx27.5\unit{k\Omega}$$
$$r_{od}\approx 2R_C=100\unit{k\Omega}$$

Common-mode:
$$r_{ic}=\frac{r_{\pi}}{2}+(\beta_F+1)R_{EE}=\frac{\beta_F}{2g_m}+(\beta_F+1)R_{EE}=\frac{100}{2\cdot7.28\unit{mS}}+101\cdot47\unit{k\Omega}\approx4.75\unit{M\Omega}$$
$$r_{oc}\approx R_C=50\unit{k\Omega}$$

\end{enumerate}

For the common-mode gain, I wrote this SPICE circuit: \\

\inputmintedindent{v}{p2_1.cir}

The result of common-mode gain is $5.25861\times10^{-1}$. \\

For the differential-mode gain, I wrote this SPICE circuit: \\

\inputmintedindent{v}{p2_2.cir}

The result of differential-mode gain is $3.47344\times10^{2}$. 

$$CMRR = \frac{3.47344\times10^{1}}{2\cdot 5.25861\times10^{-1}} = 3.30262\times10^{1}$$


\section*{Reference}
The simulation can be run with the following command: \\

\inputmintedindent{shell}{run.sh}

The results are shown below

\inputmintedindent{v}{p1_1.result}

\inputmintedindent{v}{p1_2.result}

\inputmintedindent{v}{p2_1.result}

\inputmintedindent{v}{p2_2.result}


\end{document}


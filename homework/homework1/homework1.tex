\documentclass{article}
\usepackage{enumerate}
\usepackage{amsmath}
\usepackage{amssymb}
\usepackage{graphicx}
\usepackage{subfigure}
\usepackage{geometry}
\usepackage{caption}
\usepackage{indentfirst}
\usepackage{listings}
\usepackage{xcolor}
\geometry{left=3.0cm,right=3.0cm,top=3.0cm,bottom=4.0cm}
\renewcommand{\thesection}{Problem \arabic{section}.}
\allowdisplaybreaks[4]
\lstset{
	basicstyle=\tt,
	numbers=left,
	rulesepcolor=\color{red!20!green!20!blue!20},
	escapeinside=``,
	xleftmargin=2em,xrightmargin=2em, aboveskip=1em,
	framexleftmargin=1.5mm,
	frame=shadowbox,
	backgroundcolor=\color[RGB]{245,245,244},
	keywordstyle=\color{blue}\bfseries,
	identifierstyle=\bf,
	numberstyle=\color[RGB]{20,20,20},
	commentstyle=\it\color[RGB]{96,96,96},
	stringstyle=\rmfamily\slshape\color[RGB]{128,0,0},
	showstringspaces=false,
	breaklines=true
}
\title{VE311 Homework 1}
\author{Liu Yihao 515370910207}
\date{}

\begin{document}
\maketitle

\section{}

This problem is written in C++, with CMake and the GNU Multiple Precision Arithmetic (GMP) Library in order to simplify the calculation.

For the calculation of $\pi$, Bailey-Borwein-Plouffe Formula is used
$$\pi=\sum_{k=0}^\infty\frac{1}{16^k}\left(\frac{4}{8k+1}-\frac{2}{8k+4}-\frac{1}{8k+5}-\frac{1}{8k+6}\right)$$

For the calculation of $e$, Taylor series is used
$$e=1+\sum_{k=1}^\infty\frac{1}{k!}$$

Note that the last digit is fixed in the program. The code and results are attached at the end of the report.

\section{}

According to the Peano axioms, we know these definitions of natural numbers:
\begin{enumerate}
\item 0 is a natural number.
\item For every natural number $x$, $x = x$. That is, equality is reflexive.
\item For all natural numbers $x$ and $y$, if $x = y$, then $y = x$. That is, equality is symmetric.
\item For all natural numbers $x$, $y$ and $z$, if $x = y$ and $y = z$, then $x = z$. That is, equality is transitive.
\item For all $a$ and $b$, if $b$ is a natural number and $a = b$, then $a$ is also a natural number. That is, the natural numbers are closed under equality.
\end{enumerate}

The remaining axioms define the arithmetical properties of the natural numbers. The naturals are assumed to be closed under a single-valued ``successor'' function $S$, such that $S(n) = n+1$.
\begin{enumerate}
\item For every natural number $n$, $S(n)$ is a natural number.
\item For all natural numbers $m$ and $n$, $m = n$ if and only if $S(m) = S(n)$. That is, $S$ is an injection.
\item For every natural number $n$, $S(n) = 0$ is false. That is, there is no natural number whose successor is 0.
\end{enumerate}

Addition is a function that maps two natural numbers (two elements of $N$) to another one. It is defined recursively as:
\begin{align*}
a+0&=a\\
a+S(b)&=S(a+b)
\end{align*}

Then we can demonstrate $1+1$ algebraically
\begin{align*}
1+1&=1+S(0)\\
&=S(1+0)\\
&=S(1)\\
&=2
\end{align*}

\section{}
\begin{enumerate}
\item
Since the voltage source is steady, we can treat the capacitor as open circuit and the inductor as short circuit.
$$V_L=0$$
\item
First, suppose $V_2$ doesn't exists.
$$V_0=123\angle90^\circ{\rm\,V},\omega=377{\rm\,rad/s}$$
$$i_1=\frac{V_0}{3+\frac{2j\omega(4+2/j\omega)}{2j\omega+4+2/j\omega}}$$
$$i_L=i_1\cdot\frac{4+2/j\omega}{2j\omega+4+2/j\omega}=\frac{(4+2/j\omega)V_0}{3(2j\omega+4+2/j\omega)+2j\omega(4+2/j\omega)}=\frac{(4+2/j\omega)V_0}{14j\omega+16+6/j\omega}$$
$$i_L=0.096\angle0.098^\circ{\rm\,A}$$
Second, suppose $V_0$ doesn't exists. Since the voltage source is steady, we can treat the capacitor as open circuit and the inductor as short circuit.
$$i_L'=0$$
So $$i_L=0.096\angle0.098^\circ{\rm\,A}$$
\end{enumerate}

\newpage
\section*{Attachments}

\begin{align*}
\pi=\input{ex1/pi.txt}
\end{align*}
\begin{align*}
e=\input{ex1/e.txt}
\end{align*}

\paragraph{CMakeLists.txt}\ 
\lstinputlisting{ex1/CMakeLists.txt}

\paragraph{main.cpp}\ 
\lstset{language=C++}
\lstinputlisting{ex1/main.cpp}

\end{document}

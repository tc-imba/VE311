\documentclass{article}
\usepackage{enumerate}
\usepackage{amsmath}
\usepackage{amssymb}
\usepackage{graphicx}
\usepackage{subfigure}
\usepackage{geometry}
\usepackage{caption}
\usepackage{indentfirst}
\geometry{left=3.0cm,right=3.0cm,top=3.0cm,bottom=4.0cm}
\renewcommand{\thesection}{Problem \arabic{section}.}
%\allowdisplaybreaks[4]
\newcommand{\Omegacm}{{\rm\,\Omega\cdot cm}}
\newcommand{\unit}[1]{{\rm\,#1}}

\title{VE311 Homework 2}
\author{Liu Yihao 515370910207}
\date{}

\begin{document}
\maketitle

\section{}
$$\rho=2.83\times10^{-6}\Omegacm<10^{-3}\Omegacm$$

So pure aluminum should be classified as conductor.

\section{}
$$j_n^{drift}=Q_nv_n=(-qn)(-v)=1.6\times10^{-19}\cdot10^{18}\cdot10^7=1.6\times10^6\unit{A/cm^2}$$
$$j_p^{drift}=Q_pv_p=(+qp)(+v)=1.6\times10^{-19}\cdot10^{2}\cdot10^7=1.6\times10^{-10}\unit{A/cm^2}$$
$$J=j_n^{drift}+j_p^{drift}\approx1.6\times10^6\unit{A/cm^2}$$

\section{}
$$n_i=\sqrt{BT^3\exp\left(-\frac{E_G}{kT}\right)}$$

\begin{enumerate}[(a)]
\item
$$Si:\ n_i=\sqrt{1.08\times10^{31}\cdot77^3\cdot\exp\left(-\frac{1.12}{8.62\times10^{-5}\cdot77}\right)}=5.068\times10^{-19}\unit{cm^{-3}}$$
$$Ge:\ n_i=\sqrt{2.31\times10^{31}\cdot77^3\cdot\exp\left(-\frac{0.66}{8.62\times10^{-5}\cdot77}\right)}=2.625\times10^{-4}\unit{cm^{-3}}$$
\item
$$Si:\ n_i=\sqrt{1.08\times10^{31}\cdot300^3\cdot\exp\left(-\frac{1.12}{8.62\times10^{-5}\cdot300}\right)}=6.725\times10^{9}\unit{cm^{9}}$$
$$Ge:\ n_i=\sqrt{2.31\times10^{31}\cdot300^3\cdot\exp\left(-\frac{0.66}{8.62\times10^{-5}\cdot300}\right)}=2.267\times10^{13}\unit{cm^{-3}}$$
\item
$$Si:\ n_i=\sqrt{1.08\times10^{31}\cdot500^3\cdot\exp\left(-\frac{1.12}{8.62\times10^{-5}\cdot500}\right)}=8.363\times10^{13}\unit{cm^{-3}}$$
$$Ge:\ n_i=\sqrt{2.31\times10^{31}\cdot77^3\cdot\exp\left(-\frac{0.66}{8.62\times10^{-5}\cdot77}\right)}=8.036\times10^{15}\unit{cm^{-3}}$$
\end{enumerate} 

\section{}
$$\rho=\frac{1}{\sigma}=\frac{1}{q(n\mu_n+p\mu_p)}=\frac{1}{1.6\times10^{-19}\cdot n_i\cdot(2000+750)}=10^5\Omegacm$$
$$n_i=\sqrt{BT^3\exp\left(-\frac{E_G}{kT}\right)}=2.273\times10^{10}\unit{cm^{-3}}$$
$$T=316.6\unit{K}$$

\section{}
Since $N_A>N_D$, and $N_A-N_D\gg 2n_i$,
$$p\approx N_A-N_D=4\times10^{16}\unit{cm^{-3}}$$
$$n=\frac{n_i^2}{p}=2.5\times10^{5}\unit{cm^{-3}}$$

\section{}
\begin{enumerate}[(a)]
\item Since Ge has one more electron than In, it behaves as a donor impurity.
\item Since Ge has one less electron than P, it behaves as an acceptor impurity.
\end{enumerate}

\section{}
$$N=nV=10^{16}\cdot0.5\times10^{-4}\cdot5\times10^{-4}\cdot0.5\times10^{-4}=12500\unit{atoms}$$

\section{}
$$\rho=\frac{1}{\sigma}=\frac{1}{q(n\mu_n+p\mu_p)}$$

Since we want to produce extrinsic silicon with a higher resistivity than that of intrinsic silicon, we should let
$$n\mu_n(N_T)+p\mu_p(N_T)<n_i[\mu_n(0)+\mu_p(0)]$$

Suppose $N_T=N_A$, we can get $n=p=n_i$. So the equation above can be simplified as
$$\mu_n(N_T)+\mu_p(N_T)<\mu_n(0)+\mu_p(0)$$

And we know that the functions $\mu_n(N_T)$ and $\mu_p(N_T)$ are decreasing when $N_T$ is increasing, which means when $N_T>0$,

$$\mu_n(N_T)<\mu_n(0)$$
$$\mu_p(N_T)<\mu_p(0)$$

So the equation always sets when $N_T>0$, conceptually. In conclusion, when $N_T=N_A>0$, it is conceptually to produce extrinsic silicon with a higher resistivity than that of intrinsic silicon.

\section{}
$$V_T=\frac{kT}{q}=8.62\times10^{-5}\cdot T\unit{(V)}$$
\begin{center}
\begin{tabular}{|c|c|}
\hline
$T$ (K) & $V_T$ (mV) \\\hline
50 & 4.3\\\hline
75 & 6.5\\\hline
100 & 8.6\\\hline
150 & 12.9\\\hline
200 & 17.2\\\hline
250 & 21.6\\\hline
300 & 25.9\\\hline
350 & 30.2\\\hline
400 & 34.5\\\hline
\end{tabular}
\end{center}

\section{}
$$j_n^{diff}=qD_n\frac{\partial n}{\partial x}=qV_T\mu_n\frac{\partial n}{\partial x}=1.6\times10^{-19}\cdot25.9\times10^{-3}\cdot350\cdot-\frac{10^{18}}{0.5\times10^{-4}}=-2.901\times10^{-4}\unit{A/cm^2}$$

\section{}
When $x=0$,
$$j_n^{drift}=q\mu_nnE=1.6\times10^{-19}\cdot350\cdot10^{16}\cdot-20=-11.2\unit{A/cm^2}$$
$$j_p^{drift}=q\mu_ppE=1.6\times10^{-19}\cdot150\cdot1.01\times10^{18}\cdot-20=-484.8\unit{A/cm^2}$$
$$j_n^{diff}=qD_n\frac{\partial n}{\partial x}=1.6\times10^{-19}\cdot350\cdot25.9\times10^{-3}\cdot\frac{10^4-10^{16}}{2\times10^{-4}}=-72.5\unit{A/cm^2}$$
$$j_p^{diff}=-qD_p\frac{\partial p}{\partial x}=-1.6\times10^{-19}\cdot150\cdot25.9\times10^{-3}\cdot\frac{10^{18}-1.01\times10^{18}}{2\times10^{-4}}=31.1\unit{A/cm^2}$$
$$J^T=j_n^{drift}+j_p^{drift}+j_n^{diff}+j_n^{diff}=-537.4\unit{A/cm^2}$$

When $x=1.0\unit{\mu m}$,
$$j_n^{drift}=q\mu_nnE=1.6\times10^{-19}\cdot350\cdot\frac{10^{16}+10^4}{2}\cdot-20=-5.6\unit{A/cm^2}$$
$$j_n^{diff}=qD_n\frac{\partial n}{\partial x}=1.6\times10^{-19}\cdot350\cdot25.9\times10^{-3}\cdot\frac{10^4-10^{16}}{2\times10^{-4}}=-72.5\unit{A/cm^2}$$


\end{document}
